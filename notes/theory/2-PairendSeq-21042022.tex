\chapter{Interpreting Pair Orientation}
READ PAIR INTERPRETATION
generally the LR orientation, two reads facing each other, always work with ILLUMINA (ignore the other)


elements to consider whtn using pair-end sequencing
\begin{itemize}
	\item pair ends relative orientation
	\item instert size length 
	\item coverage withing he aberrant region
	\item coverage outside of the aberrant region
	\item coverage at the break points
\end{itemize}

\end{itemize}

- INVERSION
A B inverted
how to understand an inversion? two reads over the breakpoint B and A. The two pairs shown in the image are actuallly the most informative, but they are not the only reads done.
local coverage in A and B become low, smaller depending on the number of copies. This is due o the fact that the region doesn't exist anymore.

-- OUTPUT IMAGE
inversion monoallelic, as coverage didn't totallly disappear.
reads are not mapped, consequently the coverage dicrease. over the break point some reads could align 

- TANDEM DUPLICATION
the coverage on the external region is on average 
on the extremes, the coverage is normal, on the middle region the coverage is multiplicated

- INVERTED DUPLICATION
similar to the case before
tiny drop due to the lack of the sequence which is instead replaced by the inverted duplication.

- DELETION
drop of coverage related dto 
observed distance of the two exteremes or looking to the coverage
small deletions instead are better viewable by using single end reads.