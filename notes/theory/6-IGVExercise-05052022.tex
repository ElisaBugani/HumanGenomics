\chapter{ghghg}
Paired-end reading, it is possible to understand if something strange happened to the sequence in the middle.
INVERSION
Drop in coverage on the sites of inversion
... (other cases already seen)

FIRST CASE chr1:11,043,245-11,061,901
Green reads
...

Healthy cells likely don't have tandem duplication. With perfect presence of a tumor sample. In this case, 1 of the alleles didn't undergo duplication, so why it's less than the 50\% 
Ploidy of cancer cells is an important characteristic
Purity of the sample is the portion of the sample that is tumor. Need to correct for this percentages.

SECOND CASE chr5:9,410,315-9,413,699
No coverage
Deletion on both the alleles, complete deletion of the portion

THIRD CASE chr7:31,576,117-31,599,940
No coverage in one part of the region
Inversion and an Hemyzigous deletion, it's not easy to understand what happened before and what after.
- 1st hypothesis: Inversiono first, Hemideletion second
Basically you have an inversion between B and F, and after the deletion of the ED portion, the other allele remains normal.

