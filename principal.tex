\input{prefix}

\title{\Huge\textbf{Computational Human genomics}}

\author{ Maurizio Gilioli \\
  % \small telegram: \href{https://t.me/GiacomoFantoni}{@GiacomoFantoni} \\[3pt]
% \small Github:
% \href{https://github.com/giacThePhantom/computational-microbial-genomics}{https://github.com/giacThePhantom/computationl-microbial-genomics}
}

% #TODO add contributors
% #TODO adjust glossary
\makeglossaries

\newglossaryentry{maths}
{
        name=mathematics,
        description={Mathematics is what mathematicians do}
}

\newglossaryentry{formula}
{
        name=formula,
        description={A mathematical expression}
}


\begin{document}
\maketitle
\tableofcontents

% \input{chapters/} 

% To input a new chapter, you have to copy and past this line
% here above, and use the name of your file without the tex extension. All the
% images included in the file should be inserted in a specific folder inside the
% chapter one.

\graphicspath{{chapters/ThebasicsImages/}}


\chapter{Introduction}

\textbf{\textit{Written by Chiara Campanelli}}

\section{Basic principles} \label{chap: Basics}

\begin{figure}[H]
  \includegraphics[width=4.32507in,height=3.86281in]{image1.jpeg}
  \centering
  \caption{}
\end{figure}


The words variations, aberrations and lesions are often interchanged.
Aberrations and lesions are mainly used for acquired lesions, instead variations
are mainly used for the inherited ones.

\hypertarget{genetic-make-up}{%
\subsection{Genetic Make-Up}\label{genetic-make-up}}

\begin{figure}[H]
  \includegraphics[width=1.97816in,height=2.50248in]{image2.jpeg}
  \centering
  \caption{}
  \label{fig: SNP}
\end{figure}


\textbf{Single Nucleotide Polymorphism} (SNP) is a sequence variation affecting
single bases (point mutations) (figure \ref{fig: SNP}).

The genomes of two unrelated individuals have about 1\% of different bases →
that percentage corresponds to the SNPs.

But looking at the \textbf{Copy Number Variants} (CNV), that difference will be
way higher than only 1\% DNA not present in only two copies, but in multiple,
single or even zero copies (hemizygous loss, homozygous loss). They are less
known as inherited type of variants because they are harder to detect and
identify, but they provide a lot of uniqueness in each of us.


\begin{figure}[H]
  \includegraphics[width=2.93475in,height=2.26167in]{image3.jpeg} 
  \centering
  \caption{}
  \label{fig: SNP2}
\end{figure}


Why are SNPs and CNVs so important?

They are responsible of human diversity → genetic changes

Hundreds of CNVs per individual and 20\% of them potentially affect protein-
coding genes

\textbf{Differences} in Genetic Make-up:

Very common variants are variants that are distributed in the population as the
common allele, so that 1/2 of the population has an heterozygous genotype at
that position, 1/4 has an homozygous genotype for one allele and 1/4 has an
homozygous genotype for the other allele.

\includegraphics[width=3.9928in,height=3.87812in]{image4.jpeg}

The \textbf{penetrance} is the proportion of individuals carrying an allele (or
a genotype) that also expresses the trait (phenotype) associated with it.
Obviously, penetrance is directly associated with the size of the effect
produced by the variant.

The \textbf{allele frequency} is calculated by dividing the number of times the
allele of interest is observed in a population by the total number of copies of
all the alleles at that particular genetic locus in the population.

The allele frequency is low with very rare variants

Well known variants: BRCA1/2, HOXB13, NBS1, PALB2, CHEK2 → they have moderate
size effects, meaning that all the people who have the variants, have the
disease


\hypertarget{differences-in-genetic-make-up-example}{%
\subsubsection{Differences in genetic Make-Up,
example}\label{differences-in-genetic-make-up-example}}
% article to be read? #TODO


Absorption, distribution, metabolism and elimination (ADME) genetic variants
determine pharmacokinetic variability of certain compounds, influencing the
patients' treatment response. Both common and rare variants are involved.

\includegraphics[width=6.18852in,height=2.6224in]{image5.jpeg}

Three main ways to study genetic variants: 

\begin{itemize}
  \item GWAS (genome-wide association study)
  \item SNPs panel
  \item Candidate SNP
\end{itemize}




For example, in terms of hypothesis, if I study all the variants in the human
genome and I query them in a large population, I generate data without
specifying SNP to search. Instead, if I have a very specific hypothesis, for
example I want to query if a SNPs in the CYP gene relates to the conversion of
androgens to estrogens, I don't need to run an GWAS or a wide SNPs panel. I can
query those SNPs because I have an \emph{a priori} hypothesis and I want to test
them.

This type of differential design for an experiment it is not only true for
inherited variants and ADME genes, but also to predisposition to diseases and to
study human tumors.

Precision medicine → treatment (or dosage) of a patient based on their
individual traits: takes into consideration genetic and genomic of the
individual and tumor/disease cells

\begin{itemize}
  \item Drink beer and turn red → ADME gene
  \item Athletes with a deletion of a gene, the steroids were not found in the
  anti-doping tests
\end{itemize}
 

\hypertarget{acquired-dna-aberrations}{%
\subsection{Acquired DNA aberrations}\label{acquired-dna-aberrations}}


Somatic variants are the variants NOT inherited from parents and not transmitted
to offspring. They are:

\begin{itemize}
  \item \textbf{Single Nucleotide Variants} (SNV) are somatic changes of single
  nucleotides present in only certain cells, instead of SNPs that are present in
  all cells of our body.
  \item \textbf{Indels} are changes that involve few nucleotides by INsertion
  and DEletion
  \item \textbf{Rearrangements} are mutations that can involve events like
translocations, inversion, chromothripsis,.. usually these events are caused by
breakage in the DNA double helices a two different locations, followed by a
rejoining of the broken ends to produce a new chromosomal arrangement of genes,
different from the beginning

  \item \textbf{Somatic copy number aberrations} (\textbf{SCNA}) are somatic
changes similar to CNVs. They can be every change related to the number of
copies like loss of a portion of a genome, loss of both alleles, extra copies...
\end{itemize}

\begin{figure}[H]
  \includegraphics[width=6.1748in,height=2.09646in]{image6.jpeg}\\
  \centering
  \caption{translocations}
  \label{fig: translocations}
\end{figure}

Rearrangements include:

\begin{itemize}
  \item \textbf{Balanced translocation} (figure \ref{fig: translocations}): you conserve the quantity of DNA, there
  isn't any loss or gain.
  
  \item \textbf{Unbalanced translocation}: A genomic portion is translocated
  from a chromosome to another, there is not vice versa.
  
  \item \textbf{Inversions} in only ONE chromosome: everything is normal instead
  in the break points.

  \begin{figure}[H]
    \includegraphics[width=6.22722in,height=2.39062in]{image7.jpeg}
    \centering
    \caption{Duplications inversions and deletions}
    \label{fig: Duplications inversions and deletions}
  \end{figure}

  \item \textbf{Copy number changes:} duplication or deletion. It could happen in
  the same chromosome but also in different chromosomes.
\end{itemize}

Other important modifications:

\begin{itemize}
  \item \textbf{Chromoplexy}: a class of complex somatic DNA rearrangements
  whereby abundant DNA deletions and intra- and inter-chromosomal translocations
  that have originated in an interdependent way occur within a single cell
  cycle.

  \item \textbf{Chromothripsis}: a clustered chromosomal rearrangement in
  confined genomic regions that results from a single catastrophic event,
  usually limited to one chromosome.

  \item \textbf{Kataegis}: a phenomenon that is characterized by large cluster
  of mutations (hypermutation) in the genome of cancer cells. An APOBEC family
  enzyme might be responsible fo the kataegis process.
\end{itemize}
  
\begin{figure}[H]
  \includegraphics[width=3.81944in,height=3.21433in]{image8.png}\\
  \centering
  \caption{Chromoplexy}
  \label{fig: Chromoplexy}
\end{figure}


When an aberration (clonal) occurs, all the cells will harbour the aberration
and at some point another aberration (subclonal of the other) could appear in
just one cell line. The \textbf{clonal} aberration is present in all the cells,
the \textbf{subclonal} aberration is inherited in just one cell line. Clonality
is an important information that allow us to study evolution.


\hypertarget{experimental-approaches}{%
\section{Experimental approaches}\label{experimental-approaches}}


Experimental techniques to detect variants/aberrations \textbf{prior to NGS}: a
failure because it was very hard to determine the starting points of the
aberrations. %#TODO which tecniques? 

\begin{figure}[H]
  \includegraphics[width=6.18343in,height=3.84729in]{image9.jpeg}
  \centering
  \caption{Meyerson et al. 2010, ``Advances in Understanding Cancer Genomes
  through Second-Generation Sequencing.'', Nature Reviews Genetics,
  \url{https://doi.org/10.1038/nrg2841}}
  \label{fig: DNA variants}
\end{figure}

Bulk of tumor tissue/cells from the blood procedure (figure \ref{fig: DNA variants}):


\begin{enumerate}
\def\labelenumi{\arabic{enumi})}
\item
  DNA isolation.
\item
  Gene-specific oligonucleotides (\textbf{baits}) that get hybridized onto the
  tumor DNA → the baits have a tag that allows them to be isolated.
\item
  The DNA does get fragmented.
\item
  The captured DNA is eluted and prepared into sequencing libraries.
\item
  Sequencing.
\item
  Aligned to the bait sequences.
\end{enumerate}


We repeat the procedure for healthy cells of the same individual in order to
\textbf{detect somatic mutations}.

\begin{figure}[H]
  \includegraphics[width=6.14828in,height=5.07625in]{image10.jpeg} \label{fig:
  DNA cancer variants detection}
  \centering
  \caption{Beads capture}
\end{figure}


We sequence baits because is way cheaper (exons of 50 bases instead the whole
genome)

After fragmentation procedure, before adding the adapters, we can choose between
two different sequencing approaches \ref{fig: sequencing methods}:\\


\includegraphics[width=0.75\textwidth]{image11.png}\\ \label{fig: sequencing
methods}
\includegraphics[width=0.75\textwidth]{image12.jpeg}

\begin{itemize}
  \item \textit{\textbf{Paired End (PE) sequencing}}

  You will sequence only one part of a molecule (length of 150 bp → based on the
  power of the sequencing machine we are using). You will know exactly 150 bp
  for every molecule you sequence, but you lose information (the second end of
  the pair).

  \item \textit{\textbf{Single End (SE) sequencing}}
  
  You information about the length of the DNA portion between the ends. It's
  more expensive, but:

  \begin{itemize}
    \item
      it gives information about the localization of the molecule
    \item
      you can treat each end as single read
    \end{itemize}

\end{itemize}


\subsection{Information after reads mapping over reference genome}

\begin{figure}[H]
  \includegraphics[width=6.16003in,height=3.4875in]{image13.png}
  \centering
  \caption{In the following picture: a view of reads that are mapped against
  reference genome and what we would look if we have any of the variations that
  we mentioned}
\end{figure}

Following the mapping of the reads over the reference genome, different types of
genomic alteration/information can be detected:

\begin{itemize}
  
  \item You can clearly identify \textbf{point mutations}. If a point mutation
  is present in the molecule that you sequenced and present on both alleles of
  the genome, it can be seen in all the reads very clearly.
  
  \item You might see \textbf{indels} (shown here by a dashed line). You will
  see a little space because the reference genome has more nucleotides than the
  sequenced molecule.
  
  \item If you have \textbf{homozygous deletion}, you don't see anything mapped
  in that portion: there's no DNA. Doesn't matter if SE or PE.
  
  \item If you have \textbf{hemizygous deletion}, you see the read mapped to
  that portion where the hemizygous deletion is sitting, that is more or less
  proportional to half of the reads that you have in regions where you don't
  have a copy number change. Doesn't matter if SE or PE.
  
  \item If you have \textbf{gain}, what you get is higher number of reads
  aligned against that part of DNA, underlying the fact that the molecule you
  sequenced has extra DNA for that portion of reference genome. Doesn't matter
  if SE or PE.

  \item \textbf{Translocation breakpoint} are very important!! You will have one
  end mapping the chr1 and the other end mapping the chr5. Those two ends come
  from the same molecule of the \emph{target cell} (the cell we sequenced), it
  means the cell has a translocation between chr1 and chr5. Without the PE
  protocol you cannot have this result.
  
\end{itemize}

\begin{figure}[H]
  \includegraphics[width=0.4\textwidth]{image14.png}
  \centering
  \caption{View of sequence alignment} \label{fig: coverage and allele
  frequency}
\end{figure}


The \textbf{local coverage} (cov) as shown in figure \ref{fig: coverage and
allele frequency} at position $i$ is the number of reads that span $p_i$.

The \textbf{allelic fraction} (AF) as shown in figure \ref{fig: coverage and
allele frequency} at position $i$ is the proportion of reads that supports the
reference base in $p_i$ (= the reference or the alternative allele).

\subsection{Whole Genome Sequencing Coverage}

\begin{equation} \label{eq: coverage formula}
  cov = \frac{L \dot N}{G}
\end{equation}

where:

\begin{itemize}
\item
  \textbf{L} is the read length.
\item
  \textbf{N} is the number of mapped reads.
\item
  \textbf{G} is the haploid human genome length.
\end{itemize}


This is super important because it saves us time and money when we design an
experiment. When you design an NGS experiment, you should know before what is
the type of coverage you need to answer the question you wanna ask with your
experiment. For example, if you want to look at the genotype of SNPs (inherited
polymorphisms at single side), you don't really need a coverage which is above
$10$ or $15$. So you can design your experiment in order to have an average
coverage equal to $10$ or $15$. To do that, you reverse the equation and count
how many reads you need to generate to achieve that goal. \\

\textit{N.B.}: The number of mapped reads will be always lower of the number of
reads generated by the machine (than the expected). There might be duplicates
that you might not be able to use because there might be reads that have a
quality below the threshold you intend to use.


\hypertarget{difference-between-sequence-coverage-and-physical-coverage}{%
\subsubsection{Difference between sequence coverage and physical
coverage}\label{difference-between-sequence-coverage-and-physical-coverage}}


A graphic view of how \textbf{SE (Single End Sequencing)} or \textbf{PE (Paired
End Sequencing)} can be used:

\begin{figure}[H]
  \includegraphics[width=0.65\textwidth]{image15.png}
  \centering
  \caption{Panel A - SE protocol; Panel B - PE protocol; Panel C - PE protocol}
  \label{fig: physical coverage vs coverage}
\end{figure}

Three different scenario are depicted that vary in the length of the DNA
fragments that are sequenced. \textbf{Sequence coverage} represents the number
of sequenced reads that cover the site; this affects the ability to detect point
mutations. \textbf{Physical coverage} measures the number of fragments that span
the site; this affects the ability to detect the rearrangements, based on paired
reads that map to different chromosomes. It is a way informative type of
coverage: for instance for translocations, deletions $\dots$

In Paired End sequencing protocols, the physical coverage is always higher than
the sequence coverage. Choosing the method illustrated in panel 3 (figure
\ref{fig: physical coverage vs coverage}).

Making estimation of intended coverage and observed coverage is very important.
Below I will report an example:\\


\noindent \underline{\textbf{Example coverage observation:}}\\

In these panels were designed to sequence a set of $10$ genes that the
researchers were interested in for prostate cancer. They designed this panel,
sequenced cell lines on this panel and observed the following points

\begin{itemize}
  \item On $x$-axis: the genomic location
  \item On $y$-axis: the local coverage (amplicon median coverage = each bar
  represents the local coverage of about 30 bp)
\end{itemize}

The different colors represent the different genes


\begin{itemize}
  \item \textbf{$1^{st}$ panel}: Local coverage (pile-up) of selected areas
  (targeted sequencing assay): 7 genes

  \begin{itemize}
    \item + 1 multi-gene region (T2ERG). Alternate colors indicate targeted
    areas The barplot show a single sample (LnCaP cell line; cancer cell line)
    data.
    \item Apparent \textbf{deletion} of PTEN (monoallelic deletion) because the
    local coverage of PTEN is significantly lower than the one from other genes.
  \end{itemize}
  
  \includegraphics[width=6.13481in,height=1.52625in]{image16.png}

  \item \textbf{$2^{nd}$ panel}: Local coverage (pile-up) of selected areas
  (targeted sequencing assay): 7 genes

  \begin{itemize}
    \item + 1 multi-gene region.
    \item Monoallelic deletion and partial biallelic deletion of PTEN because
    one portion is deleted and one not. PTEN has a \textbf{partial homozygous
    deletion}.
    \item The PC3 cell line shows a little bit of gain in the gene SPOP and
    FOXA1.
    \item The average coverage for the PC3 cells is approximately the same as
    the previous sample.
  \end{itemize}
  
  \includegraphics[width=6.15327in,height=1.09125in]{image17.png}

  \item \textbf{$3$rd panel}: 
  
  \begin{itemize}
    \item There's no homozygous deletion but has a high level
    \textbf{amplification} of one area \emph{absorbs} all the reads. Massive
    amplification of the Androgen Receptor (AR) → error: because it inhibits the
    sensitivity of detecting copy number changes in any other gene, the signal
    is excessively intense.
    \item When designing a panel we must pay attention and make sure that we
    don't have potential aberration that basically will draw all the attention
    of your experiment and leave you without information or \textbf{sensitivity}
    in all other regions.
    \item  It's easy to \underline{increase the experimental coverage (i.e. the
    sequence depth) at later point}. Provided tour original sample/library is
    still available, you can perform another run of sequencing and then combine
    the output from different runs.
  \end{itemize}

  \includegraphics[width=6.17134in,height=1.51875in]{image18.jpeg}
\end{itemize}


\hypertarget{note-that-this-isnt-possible-with-array-based-technologies.}{%
\subsubsection{Note that this isn't possible with array-based
technologies.}\label{note-that-this-isnt-possible-with-array-based-technologies.}}


What are the limiting factors of NGS DNA-seq experiment, in any?

Repeated regions due to \textbf{short reads}

What is the problem of short sequencing on long genome?

\begin{itemize}
  \item Complexity regions
  \item GC content
\end{itemize}
% #TODO non ho capito questo commento 


\hypertarget{the-reference-sequence-of-the-human-genome}{%
\section{The reference sequence of the human
genome}\label{the-reference-sequence-of-the-human-genome}}


Many years ago, some people claimed that the entire human genome was sequenced
but it wasn't true at all. There were still unknown or missing regions. In 2022
we finally have the complete human reference genome sequence.

But we need to consider the polymorphisms, there is no \textbf{unique} genome.
How to integrate them into a single reference genome? There is a consortium that
deals with these problems. They assemble a reference genome that reflects the
most common (in the whole population) sequences at each position of the human
genome, but also tracks information of everything that is polymorphic. So that
we can use the latest release of what they built as reference genome and then
use databases to learn about all the polymorphic sites and all the features of
every polymorphic variants.

Genome Reference Consortium:
(\href{https://www.ncbi.nlm.nih.gov/grc/human}{Genome Reference Consortium
link}) where you can find different versions of human reference genome

UCSC Genome Browser on Human: (\href{http://genome-euro.ucsc.edu/}{UCSC link})

where you can upload different versions of the reference


\hypertarget{interpreting-pair-orientation}{%
\subsection{Interpreting pair orientation}\label{interpreting-pair-orientation}}


Using IGV (Integrative Genomics Viewer) (see chapter \ref{chap: IGV} for more
details)

\includegraphics[width=6.32631in,height=4.35875in]{image19.jpeg}

The main characteristic of IGV is that it is a main view viewer: all the
information are in one window.

On the x-axis there's the genome coordinates at the top, the reference genome at
the bottom (we can select the reference genome we prefer).

Along with the data tracks there is the local coverage of the kb shown in the
window (of the sample we are looking at).

You can get any information you want of any single read that you are uploading,
very useful to see difference from the reference genome because every aberration
or whatsoever is highlighted by a different color in the local coverage of a
nucleotide base. Moreover, it gives information about the quality of the read
and the bases, if you have a PE protocol, it tells you also information about
the PE for each of them.

The \textbf{orientation} of paired end can be used to detect structural events,
including: 

\begin{itemize}
  \item Inversions
  \item Duplications
  \item Translocations
\end{itemize}


\includegraphics[width=3.89062in,height=2.65in]{image20.jpeg}


\hypertarget{inversion}{%
\subsection{Inversion}\label{inversion}}


A segment of DNA is inverted\\


\includegraphics[width=5.85745in,height=2.45625in]{image21.jpeg}


The most important pairs are the ones that stand between junctions because they
are the most informative ones.

Here one end mapped where it was on the reference genome while the other end
reversed its orientation.\\

In IGV:\\


\includegraphics[width=6.29861in,height=4.32833in]{image22.jpeg}


Information that help us:

\begin{itemize}
  \item The \textbf{insert size} from the target molecule (= the subject) is way
  longer. For all the pairs that are at the breakpoint, the insert size is
  different from the expected.

  \item The \textbf{orientation} is different.
  
  \item If you look at the local coverage, you can see a \textbf{drop} in two
  points: at the breakpoints. The reads that are mapping the junctions cannot
  map the reference genome because the breakpoint sequence is altereed in the
  reference genome. So, if we have an inversion in only one of the two alleles,
  then the reads coming from the allele with the inversion will not contribute
  to the local coverage at the breakpoint. The contribution to the local
  coverage will come only from one allele.
\end{itemize}


Moreover, we can notice that the coverage on the middle part does not change
significantly from the coverage on the sides.

When you align reads against a genome, you can \underline{allow for a certain
mismatches or partial alignment}. So, if you impose certain thresholds to your
aligner, you can also say that if there are reads that align for 80\% and have
20\% of sequences misaligned, you align them in any case. So you will have reads
that are correct up to the breakpoints and the browser will shows the mismatches
beyond the breakpoint. So, you can have a partial drop of coverage because you
allow mismatches in your alignment.


\hypertarget{tandem-duplication}{%
\subsection{Tandem duplication}\label{tandem-duplication}}


A segment of DNA is duplicated and inserted in the target molecule adjacent to
the original one.


\includegraphics[width=6.1113in,height=2.55073in]{image23.jpeg}\\

So, as result, the orientation instead of going inward goes outward.\\

\includegraphics[width=6.22598in,height=4.14375in]{image24.jpeg}\\

\emph{What do you expect to see from coverage?} We will have a gain in coverage
that is proportional to the extra copy. We need to pay attention to the double
because it is a double contribution of that allele, but if a tandem duplication
happens only in one allele and the other allele has his own one copy, then the
local coverage corresponding to the tandem duplication will be 3/2 of the
expected coverage.\\

If you have a read that maps BA, do you expect to see it in the mapped reads?
Partial mapping. As we said before, if you allow your mapper to have some
mismatches of a certain percentage of bases from your reads, you can still see
some coverage contributed on one end of the segment and mismatches on the other
side.
% #TODO da riscrivere

For what concern the \underline{junctions}, you shouldn't see any difference of
coverage because that sequence exists only once in the target molecule. The
local coverage increases only in correspondence of the segment AB.


\hypertarget{inverted-duplication}{%
\subsection{Inverted duplication}\label{inverted-duplication}}


The duplication is inverted but it's not located near the original fragment, but
somewhere else.

\includegraphics[width=6.11483in,height=2.82187in]{image25.png}\\


\includegraphics[width=6.29059in,height=4.34375in]{image26.jpeg}\\


There is a gain of coverage in the duplicated region and a tiny drop in the
break points where the sequence exists in only one allele.


\hypertarget{deletion}{%
\subsection{Deletion}\label{deletion}}


Deletion of a segment of DNA:

\begin{itemize}
  \item If the deletion \underline{is larger} than the size of the reads, we
  should see half of the coverage in the deleted regions.
  \item If the deletion \underline{is shorter} that the size of the reads, we
  should see a tiny little space corresponding to the missing nucleotides.

\end{itemize}




\input{chapters/Genetic_Fingerprinting}
\input{chapters/IGV}
\graphicspath{{chapters/TumorEvAndVesciclesImages/}}


\chapter{Tumor evolution studies via NGS data}


\section{Why studying tumor evolution?}

  Understanding tumor evolution is useful for: 
  \begin{itemize}
    \item \textbf{Academical purpose}; mainly for research
    \item \textbf{Clinical purpose}; the order of somatic events during tumor evolution can be relevant when considering the management of a patient, e.g. it can affect the treatment decided by the tumor board. A \textbf{tumor board} is an organism present mostly in research oriented ospitals (but also non research oriented ones and its popularity is increasing), and it is composed by different specialists (oncologist, genetist, radiologist, comutational biologist, pathologist and others) which manage the patients jointly. Aside from clinical purposes, this organism is also useful for training new experts. 
  \end{itemize}


\section{Tumor heterogeneity}
  
  A tumor can arise due to:
  \begin{itemize}
    \item A single (or few) strong driver mutation (in oncogenes or oncosoppressor genes)
    \item Several mutations that gradually change the cell phenotype without leading to cell death
  \end{itemize}
  In both cases the mutations are somatic events due to stochastic processes, mainly due to carcinogenic substances that damage DNA therefore causing mutations (but also physcal phenomena such as radiations and others). These mutations are mostly associated to cell growth; this is why tumor cells often undergo clonal expansion and create a mass. The speed at which the mass grows and mutates is dictated by the mutations that occurred. 
  
  A tumor mass can be either homogeneous or heterogeneous; in general, the more aggressive and old the tumor is, the higher the degree of heterogeneity. Higher heterogeneity usually correlates to drug resistance (since some of the clonal populations mught be able to better resist the drug compared to others).

  New Generation Sequencing allows to study all the somatic mutations that occurred in a cell, both the cancer related ones and the benign ones. By sequencing with an appropriate depth, you can infer in which fraction of the cell population a certain mutation is present; this allows to reconstruct the clonality of the tumor and the mutation history. Notice that since very deep sequencing can give errors, you usually need to check different loci in order to consolidate your result. 

  Tumor heterogeneity can be subdivided into: 
  \begin{itemize}
    \item \textbf{Inter-individual heterogeneity}: tumor from patient A is different from that of patient B
    \item \textbf{Intra-individual heterogeneity}: tumors from the same patient might differ
    \begin{itemize}
      \item \textbf{Spatial heterogeneity}: synchronous tumor masses in the same patient might differ 
      \item \textbf{Temporal heterogeneity}: a tumor might changes overtime, due to spontaneous or drug induced selection
      \item \textbf{Intra-lesion heterogeneity}: an individual tumor mass might present different lesions (which display different clones with different mutations, therefore different phenotypes, treatment resistances and so on)
    \end{itemize}
  \end{itemize}
  Almost always, many of these heterogeneities are present simultaneously. 
  
  Notice that genetic heterogeneity does not necessarely reflect morphological heterogeneity (e.g. different prostate lesions might look the same when stained but then display different markers using in situ immunochemistry). Moreover tumor mass size does not necessarely correlate with aggressiveness (hence imaging is not enough to study tumors).
  Heterogeneity might cause problem in the interpretation of the spectrum obtained via sanger sequencing, since the sample might contain different sequences for the same locus, hence leading to an overlap in the peaks. 
  In case of different lesions, we can define tumor burden and features of each of them via individual sequencing.

  Tumor evolution can happen in two ways:
  \begin{itemize}
    \item \textbf{Linear evolution}: genetic instability leads to new tumor clones and if those display some advantage compared to the previous ones, the older ones get replaced by the new ones (otherwies the new clone dies down). In this case you generally have low heterogeneity. 
    \item \textbf{Branched evolution}: genetic instability leads to the formation, from an ancestral clone, of different clonal populations which can coexist in the same or different tumor masses. In this case you generally have high heterogeneity.
  \end{itemize}
  
  A metastasis can either have:
  \begin{itemize}
    \item \textbf{Monoclonal origin}: meaning that it originates from tumor cells coming from a single lesion. In this case you have similar features as the starting mass and overall low heterogeneity within the metastasis.
    \item \textbf{Polyclonal origin}: meaning that it originates from multiple tumor cells coming from different lesions. This phenomenon is called \textbf{multiclonal seeding} and it leads to high lesion heterogeneity. Moreover, the fact that it displays some of the features from each of the parental lesions makes the analysis more complex.
  \end{itemize}
  
  As previously mentioned, tumor heterogeneity plays a big role in defining treatment resistance.
  We talk about two types of drug resistance:
  \begin{itemize}
    \item \textbf{Primary resistance}: the pre-treatment tumor mass already contains cells that are resistant to the treatment; the treatment kills the non-resistant cells, hence the resistant clone expands. 
    \item \textbf{Acquired resistance}: the pre-treatment tumor mass does not already contains cells completely resistant to the treatment; the clones that can survive the treatment the best could then mutate in order to acquire a treatment immunity mechanism. 
  \end{itemize}
  In case of primary resistance, the tumor might already display some biomarkers pointing to some treatment resistance; this is useful for the tumor boards in order to avoid needless harmful treatments. However, no biomarkers for each treatment are known, plus the tumor can always evolve unpredictably and acquire a new resistance. 


\section{Algorithms to study tumor evolution}
  
  You can study tumor evolution using information from:
  \begin{itemize}
    \item Samples from the \textbf{same subject}, from different time points or lesions; this way you can reconstruct mutation order and metastatic processes within the individual (base on shared or not mutations). 
    \item Samples from \textbf{different subjects} affected by the same pathology (e.g. prostate cancer); you use recurring patterns across individuals, this way you can reconstruct more generic features of the pathogenesis, for instance:
    \begin{itemize}
      \item Very common mutations in the pathology (those shared across many individuals)
      \item Mutations that tend to happen in a specific order (take for instance two mutations \textit{A} and \textit{B}; if in the majority of tumors which present both lesions, \textit{B} is almost always subclonal to \textit{A}, then probably \textit{A} tends to happen prior to \textit{B}).
    \end{itemize}
  \end{itemize}

  For more in depth reading (clickable links):
  \begin{itemize}
    \item \href{https://pubmed.ncbi.nlm.nih.gov/25830880/}{\textit{The evolutionary history of lethal metastatic prostate cancer, Gundem et al, Nature 2015}}
    \item \href{https://pubmed.ncbi.nlm.nih.gov/23622249/}{\textit{Punctuated evolution of prostate cancer genomes, Baca et al, Cell 2013}}
  \end{itemize}

  \textit{NOTE}: I did not add some pictures even though they were commented in class because I think that the relevant part was understanding the points listed above. 

  In general, when you have some tumor data, you try to see which of your models best fits the progression. 

  There are several aspects that must be taken into account during this type of analysis; most of them are useful in comprehending the pathology and its mechanisms, but at the same time they make analyzing the NGS data more difficult. Some of these aspects are:
  \begin{itemize}
    \item \textbf{Heterogeneity} (inter-patient, intra-patient, intra-tumor)
    \item \textbf{Time dependence} (tumor changes overtime)
    \item \textbf{Treatment status} (was the tumor treated, if yes how?)
    \item \textbf{Admixture DNA} (presence of non-tumoral DNA, \textit{explained more in depth below})
  \end{itemize}

  In a tumor biopsy you could have (and this is generally the case) other cells that are not tumoral (healthy tissue cells, stromal cells, leukocytes...). It is then defined the concept of \textbf{admixture}, which is \textit{the fraction of non-tumoral DNA within the sample}. Admixture is then used to define \textbf{tumor purity}, which is
  $$
  \text{tumor purity } = 1 - \text{ admixture}
  $$
  To sum up, a fully tumoral sample would have admixture equal to zero and purity equal to one. The opposite holds for healthy tissue (purity equal to zero, admixture equal to one).
  Deconvoluting the sequences derived from admixed DNA complicates NGS data analysis, but also provides useful information:
  \begin{itemize}
    \item Aggressiveness of a lesion; in general, the lower the purity, the better the outcome
    \item Defining whether a mutation is actually part of a subclonal tumor population or it is just admixed DNA
  \end{itemize}
  
  The most useful feature from NGS for characterizing tumor evolution (clonality, purity and so on) are:
  \begin{itemize}
    \item Copy number mutations
    \item Point mutations
    \item Single cell sequencing
    \item Polymorphic information (which SNPs does the tumor have)
  \end{itemize}

  The algorithms used to study tumor evolution use \textbf{informative SNPs}, meaning:
  \begin{itemize}
    \item SNPs for which the individual is heterozigous (hence they vary from individual to individual)
    \item SNPs for which the allelic fraction is easily measurable 
  \end{itemize}
  Making parsimonious assumptions (mainly that all clones have the same growth rate), these algorithms allow to study any form of genetic aberration. 

  \textit{REMINDER}: Do not confuse the following concepts:
  \begin{itemize}
    \item \textbf{Minor allele frequency}: frequency of the alternative allele for a locus in the \textbf{entire population}
    \item \textbf{Allelic fraction}: frequency of the alternative allele for a locus in a \textbf{single individual}. It is a local property of the individual. In terms of NGS this becomes:
    $$
    \text{Allelic fraction} = \frac{\text{locus reads with minor allele}}{\text{total locus reads}} 
    $$
  \end{itemize} 

  Other important concepts to consider are:
  \begin{itemize}
    \item \textbf{Neutral reads}: reads equally representing parental chromosomes 
    \item \textbf{Beta fraction}: percentage of neutral reads. Beta goes from 0 to 1; the closer the value to 1, the closer the reads are to a 50/50 split among parental sequences, the closer the value to 0, the closer the reads are to a 100/0 split in favour of either parental sequence. 
  \end{itemize}
  
  \textit{NOTE}: The way to compute beta values is in the slides but skipped during the lecture 
  
  The \textbf{allelic fraction} for an informative SNP can be:
  \begin{itemize}
    \item 0 if the alternative allele was deleted
    \item 1 if the reference allele was deleted (the non-alternative one)
    \item Around 1/2 if both alleles are present in equal proportion
    \item Some other value in the range (0,1), that could be due to duplication, heterogeneity (admixture and/or subclonality), errors and so on. In this case some further information might be required (for instance the coverage)
  \end{itemize}

  The \textbf{beta fraction} can be:
  \begin{itemize}
    \item 0 if either allele was deleted (hence you have only one)
    \item 1 if both alleles are equally-represented (normal condition)
    \item Any other value in the range (0,1), and this is also due to heterogeneity and other factors.
  \end{itemize}

  Notice that allelic fraction and beta fraction give similar information, but the beta fraction is not allele specific, hence it is better suited to study genetic abnormalities. 

  Using allele frequency and beta fraction, informative SNPs can be used to reconstruct the genealogy of the mutations. If there is a deletion of a region then you have a loss of heterozygosis for all SNPs in that region (since you chose informative SNPs, hence heterozygous ones); then based on the mutations present or absent in the different clones of the lesion (since you do not have a perfectly homogeneous mass) you can reconstruct their order.  

  When designing a test you need multiple informative SNPs for each genomic fragment of interest. Moreover you have to choose alleles that have high MAF (hence the minor allele frequency is still rather high), since those are more likely to give you information. 

  For more in depth reading (clickable links):
  \begin{itemize}
    \item \href{https://pubmed.ncbi.nlm.nih.gov/31524989/}{\textit{Ploidy- and Purity-Adjusted Allele-Specific DNA Analysis Using CLONETv2, Davide Prandi, Francesca Demichelis, 2019}}
  \end{itemize}

  \begin{figure}[H]
  \includegraphics{image_01.jpg}
  \end{figure}


\section{Estimating admixture and clonality}
  Notice that the beta fraction correlates with the shape of the distribution of the allelic fractions of the informatives SNPs in the read; with beta = 1, you have a normal distribution with mean 0.5, with beta = 0 you have two sharp peaks at 0 and 1, with any intermediate value you have to peaks which can be partially overlapping for values close to 1. Notice that increasing the coverage does increase the resolution of the peaks.
  For this reason increasing the coverage (with beta constant) does increase the ability to distinguish clonality, especially of populations that are only some degree of difference from each other.
  
  \begin{figure}[H]
  \includegraphics{image_02.jpg}
  \end{figure}

  To estimate the admixture/clonality of a cell population:
  \begin{itemize}
    \item Measure the allelic fraction and beta fraction of each informative SNP of a genomic region
    \item For each region try which of the models fits your data the best (basically map the distribution of the allelic fractions of the region against prefitted reference distributions) 
    \item You can then compute the local and the global admixture:
    \begin{itemize}
      \item \textbf{Local admixture} is a measure of the fraction of cells displaying a certain lesion with respect to another; for this reason local admixture is used as an estimate for \textbf{clonality}
      \item \textbf{Global admixture} is a measure of how many cells, on average, have a lesion; this can be used to estimate the \textbf{DNA admixture} (purity) of the sample
    \end{itemize}
  \end{itemize}
  Hence this technique allows you to distinguish purity and subclonality.
  
  Graphically you obtain a plot with:
  \begin{itemize}
    \item On the x axis, the cromosomal coordinates indexed by informative SNPs. The longer the horizontal segment, the bigger the considered region. 
    \item On the y axis, the MAF values for the informative SNPs. MAF values are mirrored on 0.5, since you do not care about distinguishing the alleles. Any drop below the 0.5 value means that the region does not have a 50/50 split. The deeper the drop the deeper the difference in the representation of the alleles. 
  \end{itemize}
  In the example picture, the top subplot shows drops which have very similar depth, hence global and local admixture are similar and there is very low heterogeneity. In the bottom subplot you have differences in local and global admixture, hence we can infer the presence of different clonal populations. 

  \begin{figure}[H]
  \includegraphics{image_03.jpg}
  \end{figure}
  
  For this type of analysis is always useful to have the \textbf{match normal DNA} (the non-tumor DNA of the subject): match normal DNA is usually obtained from leukocytes in the blood, otherwhise one could somehow deconvolute the signal of the admixed cells. 

  Another graphical representation in bidimensional space is the following plot:
  \begin{itemize}
    \item On the x axis the \textbf{log2 ratio}, meaning 
      $$
      \text{log2 ratio} = \log_2\frac{\text{local tumor coverage}}{\text{local normal coverage}}
      $$ 
      This indicates how abundant cancer DNA is with respect to healty DNA (gain of DNA if above zero, loss of DNA if below zero).
    \item On the y axis the \textbf{apparent admixture}, which is defined as
      $$
      \text{Adm. apparent} = \frac{\beta}{2 - \beta}
      $$
      Notice that this measure refers to each individual deletion/abnormality.
    \item The dots which represent the individual genomic segments. The dots tend to create multiple clusters and the closer two points are, the more probable the events they represent are close to each other in time.
  \end{itemize}
  
  \begin{figure}[H]
  \includegraphics{image_04.png}
  \end{figure}
  
  You can compute clonality using the formula:
  $$
  \text{clonality} = \frac{1 - \text{Adm. apparent}}{1 - \text{Adm. global}}
  $$

  An example of how heterogeineity can lead to difficult to interpret results can be found in the following paper:
  \href{https://pubmed.ncbi.nlm.nih.gov/25160065/}{\textit{Unraveling the clonal hierarchy of somatic genomic aberrations}}
  \textit{NOTE}: The case study was rushly explained during the lecture, but in my opinion it did not provide any further information; this is one of the papers uploaded on moodle.



\input{chapters/TumorEvolutionStudiesII}
\graphicspath{{chapters/TumorEvAndVesciclesImages/}}

\chapter{Extracellular vesicles}

\section{Introduction to EVs}

  "Extracellular vesicles are \textbf{membrane-enclosed nanoscale particles} released from essentially all prokaryotic and eukaryotic cells that \textbf{carry proteins, lipids, RNA and DNA}." (\textit{"RNA delivery by extracellular vesicles in mammalian cells and its applications", O’Brien et al, 2020 - Nature Reviews Molecular Cell Biology}). The content of an extracellular vesicle (EV), including membrane proteins, is usually referred to as \textbf{cargo}.

  EVs present many different surface proteins, mainly \textbf{tetraspanins} (commonly used markers to identify them) but also receptors, adhesion molecules and immune system ligands; these molecules are responsible of the targetting function, meaning that they define which cells the EV should interact with. 

  \begin{figure}[H]
  \includegraphics[scale=0.34]{image_07.png}
  \end{figure}
   
  The cargo of an EV tends to reflect the state of the cell that produced it; this way EVs become a way to transport material from a cell to another, but mostly to comunicate even at long distances (since EVs can enter the blood stream).

  Different types of EVs exist (different cargo, dimensions, genesis...). In older literature EVs were calssified based on the cells that produced them and/or their size and/or function (e.g. large oncosomes); now the International Society for Extracellular Vesicles (ISEV) suggests to classify EVs into three categories, \textbf{exosomes}, \textbf{microvesicles} and \textbf{apoptotic bodies}. The respective characteristics are summarized in the following table.
  
  \begin{figure}[H]
  \includegraphics[scale=0.8]{image_05.png}
  \end{figure}

  Notice that size is not enough to subdivide them since there are overlaps. A way better way of classifying EVs is by their genesis:
  \begin{itemize}
    \item Exosomes originate from \textbf{multivesicular bodies}. Multivesicular bodies are organelles whose membrane buds inward creating \textbf{intraluminal vesicles}; then when the multivesicular bodies merge with the plasmatic membrane their content is released into the extracellular environment and intraluminal vesicles become exosomes.
    \item Microvesicles originate from outward budding of the plasmatic membrane.
    \item Apoptotic bodies are generated from apoptotic cells through various mechanisms.
  \end{itemize}

  \begin{figure}[H]
  \includegraphics[scale=0.5]{image_06.png}
  \end{figure}

  The EV genesis defines the cargo and the surface markers, which define the target cell for the EV. Surface markers can be used in \textbf{flow cytometry} to divide these three populations. It is important to note that the content of exosomes and microvesicles is not just a portion of the cytoplasm of the cell generating them, but it is also enriched in specific molecules thanks to \textbf{selective sorting} mechanisms (only some of which are known), especially for proteins and RNAs.

  EV uptake by target cells can occur through different modalities (depending on target cell, vesicle size and others):
  \begin{itemize}
    \item Phagoytosis 
    \item Macropinocytosis 
    \item Clatrhrin dependent endocytosis
    \item Receptor mediated endocytosis 
    \item Fusion with the plasmatic membrane
  \end{itemize}


\section{Medical applications of EVs}

  Due to the aforementioned facts that:
  \begin{itemize}
    \item Almost all cells produce EVs, cancer and other diseased cells included
    \item The cargo reflects the functional status of the secerning cell
    \item EVs can convey signal through long distances
  \end{itemize}
  EVs are being studied more and more for clinical applications, mainly to understand disease pathogenesis (and eventually how to act on it) and to potentially use them in early-screening assays. 
  The fact that tumor derived EVs play a major role in reorganizing the surrounding environment has already been proven: tumor derived EVs induce endothelial proliferation and thus neoangiogenesis, fibroblast differentiation and extracellular matrix remodeling. This last aspect is especially relevant since EVs can create niches that facilitate the metastatic process. 

  One other significant advantage of EVs is the fact that they can be obtained through liquid biopsy, together with circulating tumor cells (CTCs), cell free DNA (cfDNA) and ribolipoproteins. Liquid biopsy generally refers to blood samples, therefore a non invasive technique to study the disease.
  Integrating transcriptomic data from EVs with other techniques could provide efficient biomarkers for early detection and biomarkers to easily measure overtime to track tumor evolution or response to a treatment. 

  In order to use EVs in clinical applications, a consistent and reproducible way of extracting EVs from the liquid biopsy is needed. In the context of the PRIME project (PRostate cancer plasma Integrative Multi-modal Evaluation consortium), different methods to extract EVs from plasma (from prostate cancer and healty patients) have been tested and compared, those methods being:
  \begin{itemize}
    \item \textbf{Nickel-based isolation (NBI)}, which utilizes the fact that EVs are negatively charged, hence they are attracted by positively charged nickel beads; the beads are then retrieved using magnets and the EVs are eluted from their surface. 
    \item \textbf{Size exclusion cromatography (SEC)}, which is a chromatographic technique in which the retention time of an object in the column depends on their size (the bigger the object, the fewer the pores of the stationary phase it can enter into, the faster the object is eluted). This technique requires knowing the elution times of the EVs, which can be difficult considering their high heterogeneity.
    \item \textbf{Ultracentrifugation (UCFG)}, meaning centrifugation at around 22000 RPM for some hours. EVs are too small to sediment using regular benchtop centrifuges. 
  \end{itemize}
  The total amount of RNA obtained from each extraction was measured using SMARTseq kit. 

  The result was that different isolation methods were highly reproducible on homogeneous samples (EVs from prostate cancer cell lines culture medium), while the different methods showed higher variability on heterogeneous samples (EVs from blood). This shows that different isolation methods isolate better different EV populations, expecially in samples such as blood which contains a miriad of EVs with different origin and characteristics, since all healty tissues produce EVs that mix with tumor derived ones in the blood. This leads to the need to identify which method is the best in order to enrich tumor derived EVs rather than healthy tissue derived ones; no standard protocol for the isolation of EVs and the analysis of their transcriptome is available. 
  
  Another challenge, deriving from the use of EVs, is the fact that RNA signal deriving from multiple populations is difficult to interpret. Ideally one would need some way to deconvolute the signal into the components associated with each cell population; in order to do so, two potential approaches are possible: 
  \begin{itemize}
    \item \textbf{Supervised deconvolution}, which means using known cell line signatures to split the signal and compute the fraction of contribution for each population (one tool that does this is CIBERSORT). The main problem with this method is that it is not possible to get the signal for unknown cell populations; moreover it is difficult to define cell population specific signatures since we do not have pure EVs populations obtained from blood. 
    \item \textbf{Unsupervised deconvolution}, which means using unsupervised clustering algorithms to subdivide the signals into populations; the problem is that these approaches tend to be less sensitive and the identification of the clusters is not simple. Still, this allows to identify not previously known cell populations.
  \end{itemize}
  Deconvolution approaches are therefore plausible but not well established.

\section{EVs conference}
  Notes from the conference \textit{"Extracellular vesicles as diagnostic and therapeutic tools for kidney diseases, Benedetta Bussolati, Dept. of Molecular Biotechnology and Health Sciences, University of Torino, 12 May 2022"}.

  \textit{DISCLAIMER}: they might not be perfect but they should suffice for a general idea of the concept
  
  EVs are part of the secretome produced by stem cells in order to try and induce tissue regeneration after damage, since they do not act directly in the regeneration process by differentiating. For this reason stem cell derived EVs (scdEVs), are subject of study for potential tissue therapies. Most studies have been performed on mesenchymal stem cell derived EVs, but some studies have shown that, despite the great heterogeneity of EVs produced by stem cells, little to no difference was found between mesenchymal stem cells derived EVs and other scdEVs. Of this heterogeneous population (especially regarding the expressed tetraspanins), small EVs seem to be the ones which are more associated to tissue regeneration; moreover small EVs are the safer ones for potential medical applications since they do not display HLA or tissue factor, that are sometimes present in bigger vesicles and that could lead to immune response and coagulation respectively. A very big spectrum of genes is regulated simultaneously through EVs; to support the role of EV content in the regenerative process, DROSHA KO models (which have impaired miRNA loading into EVs) lose most of their therapeutic effect.

  One way to analyse EVs is MACSPlex, a cytofluorimetric tool with beads and detection antibodies for tetraspanins; that being said, especially for medical purposes, better and more standardized ways to quantify, identify and test the potency of EVs are still needed. Notice that the potency of an EV, meaning its ability to induce a certain response, is highly application dependent. 

  Regarding the activity of EVs on kidney diseases, there have been different studies:
  \begin{itemize}
    \item Renal damage markers in kidney injury model decrease overtime in presence of stem cells or just scdEVs; the responses in the two cases are basically the same, suggesting that most of the therapeutic effect of stem cells in kidney diseases is due to EVs.
    \item Repeated administrations of scdEVs to diabetic nephropaty models reduce inflammation and fibrosis
    \item In healhy patients most vesicles reach liver and spleen, while in diseased patients one can find more EVs than usual in the damaged site; this supports both a specific and an aspecific targetting of EVs to the damaged area. This holds true even in kidney disease models, where administered intravenous EVs reach the kidneys within 15 minutes. 
    \item A phase 1 study has been conducted on the use of scdEVs in kidney diseases. 
    \item Urine derived EVs are comparable in potency with MSCs EVs. Urine derived EVs are all generated from kidney cells, since no EVs from the blood stream can pass the glomerular filtration membrane; this also means that urine derived EVs could be used as a diagnostic tool for kidney diseases. 
    \item Klotho, a recently discovered hormone, is produced mostly by the kidney both in a transmembrane and in a soluble form. Klotho KO murine models have a significantly shorter lifespan and a faster aging process. Klotho can be found both in urine and blood. Moreover klotho has been found coexpressed with tetraspanins, confirming its presence EVs. By providing recombinant klotho EVs to a klotho KO mouse model, the healthy phenotype is rescued; moreover providing klotho through EVs rather than by direct injection is more efficient. 
    \item Autologous urinary EVs seem to have a beneficial effect in kidney injury model.
    \item CD133+ is a marker for regenerative kidney cells in adult humans; studies have tried to test if its expression on urinary EVs correlates with the outcome of kidney transplant. Healthy human urinary EVs display very high levels of CD133+. Bad responders to kidney transplant displayed lower levels of CD133+ compared to good responders, but both had significantly lower levels of expression compared to healthy individuals. Blood and urine samples of kidney transplant patients were collected at various timepoints during a period of one year. Samples were centrifuged to remove bigger debris, analyzed using MACSPlex and then normalized for the number of identified tetraspanins. Some of the patients did not recover while others did; by analyzing samples from 10 days after the transplant was possible to predict which patients would recover and which would not. Both in blood and urine EVs were found different biomarkers (among which CD133+) whose concentration was significantly higher in individuals that would recover. 
  \end{itemize}



% #TODO solve minor problems

\clearpage

\printglossary
%\bibliographystyle{apacite} \bibliography{bibliography}

\bibliographystyle{plain}
% \bibliography{bibliography.bib}


\end{document}
